\documentclass[12pt,final]{article}

\usepackage{commands}

\author{\vspace{-20pt}Benjamyn}
\date{\today}
\title{\vspace{-40pt}Terrapin Tango Festival Documentation/Readme}

\addtolength{\voffset}{-80pt}
\addtolength{\hoffset}{-60pt}
\addtolength{\textwidth}{110pt}
\addtolength{\textheight}{130pt}

\newcommand{\pspace}{\vspace{10pt}}

\begin{document}
\pdfbookmark[1]{Contents}{table}
\maketitle
\tableofcontents

\sn{Drivers}

There are three primary drivers used for processing registrants, 
performing registration activities, and sending out emails to all
the confirmed registrants. 
\benum
	\citem{PaymentValidator}{Used to look at the status of registrants in the
		Google Spreadsheet. This may include their information such as name,
		leader/follower, etc. The classes they have signed up for, total cost, and
		his/her eticket and payment status.}
	\citem{OnsiteRegisterer}{Use this program to quickly register people
		onsite. Payment is accomplished using \href{www.squareup.com}{Square}, but
		this system provides us with the attendee's class schedule and provides a
		mechanism to send an e-ticket to their email address.}
	\citem{MassEmailer}{The mass emailing system is used to send an email to
		every paid registrant in the \tb{Eary Registration} spreadsheet. Both the
		\emph{formsite} and \emph{onsite} worksheets are used to generate the list
		of email addresses that the message is sent to.}
\eenum

\sn{Configuration of the Drivers}

The configuration of registration system is completely configurable for the
most part. Here is a list (which includes their relative path location from the
project directory) and description of the relevant files:
\benum
	\bolditem{files/config.xml}{this is the top level configuration file. It contains:
		\benum
			\emitem{\tb{email information}}{there are three types of emails:
				eticket, prefestival, and thank you. The eticket email is what is
				contained in the eticket email which has the eticket attached. The
				prefesival and thank you emails are sent using the mass emailing
				system. Each email type has a configurable subject heading and an
				html file that will be put into the body of the email (open the
				file in a browser to view how it will look in a recipient's
				inbox).}
			\emitem{\tb{eticket picture file}}{location of the image used to print the eticket}
			\emitem{\tb{festival classes XML file}}{this XML file is used by the
				onsite registration system to get a list of
				classes/milongas/special passes and their respective prices and
				data. The reason this file exists is because it would be very
				difficult to parse the google spreadsheet generated by formsite.
				So, every year, both the formsite and this XML file need to be
				updated. Stop crying. It's not that bad.}
			\emitem{\tb{account information file}}{this XML file contains the gmail
				user name and password, and the google spreadsheet information.}
			\emitem{\tb{tax percent}}{this is the ``processing'' fee that is
				included into a festival pass to offset the processing cost of
				using square to charge customers.}
		\eenum
	}
	\bolditem{files/info/accountInfo.xml}{XML file containing the account
		information including the google username and password as well as the
		google spreadsheet information that formsite dumps its registration
		information to.}
	\bolditem{files/info/festivalClasses.xml}{meta-data including times/artists
		of classes, milongas, and pricing information used for the onsite
		registraiton system.}
	\bolditem{files/media/ticketEmail.html}{body of the eticket email written in html}
	\bolditem{files/media/prefestivalEmail.html}{body of the prefestival email written in html}
	\bolditem{files/media/thankyou.html}{body of the thank you email written in html}
\eenum

\ssn{ETicket Picture}

Just wanted to let you know that the eticket PNG file and GIMP source is found in\\
\code{files/media/tango-eticket-2013.png} and 
\code{files/media/tango-eticket-2013.xcf}. Have fun Betsy.

\ssn{Further Configuration: small projection for someone?}

Further configuration is usually not needed, but if the formsite questions
change, then this will impact how they are laid out in the google spreadsheet.
The columns of the different pieces of data in the spreadsheet are hardcoded
right now. Yes, I know, it's not the best, but I'm too lazy to change it to an
XML configuration file. Here's an excellent project for someone.

The two relevant files are:
\benum
	\citem{src/datastructures/Constants.java}{Contains the column numbers for
		the different pieces of information in the formsite google spreadsheet. It
		also contains an array of strings that are used to filter the list of
		classes with. For example, if a registrant chooses ``none'' for a class
		period, we don't want to print ``none'' on their e-ticket. So, we remove
		these strings from the list returned from the google spreadsheet.}
	\citem{src/datastructures/Enums.java}{I doubt you will need to modify this
		file. It contains some information on XML tags, but also contains the
		functions for converting the strings to enumeration types e.g. \code{"umd"}
		to the enumerator type \code{enum StudentType.UMD\_GMU\_STUDENT}. It also
		contains the student type strings used in the eticket.}
\eenum

\sn{Driver Arguments: MassEmailer}

The \code{PaymentValidator} and \code{OnsiteRegisterer} do not take any
arguments. However, the \code{MassEmailer} requires a single argument
representing which email type to send i.e. to send the \emph{prefestival
information} email or the \emph{thank you} email. The email is sent to all of
the \tb{paid} registrants in both the early and late spreadsheets and the
respective \emph{formsite} and \emph{onsite} worksheets within those
spreadsheets.

\sn{Producing a Java Executable}

To produce a java executable using eclipse, go to:
\code{File$\rightarrow$Export}. Choose \emph{Runnable JAR file} from the
\emph{Java} folder. The \emph{Launch Configuration} is the name of the class
that contains the driver. This will be \code{PaymentValidator},
\code{OnsiteRegisterer}, or \code{MassEmailer}. You should chose the option
\emph{Package required libraries into generated JAR} so that the program is
completely self-inclusive. Click \emph{Finish} and you're done!

There might be some warnings, but don't worry about them. You can then run the
jar using Java. 

\ssn{Running in Windows}

In Windows, the \code{PaymentValidator} and
\code{OnsiteRegisterer} can be run by just double clicking on the jar file.
The \code{MassEmailer} is a bit more difficult to run in Windows because it
takes an argument, so you need to run this using DOS/Command Prompt. I don't
know where it is in Windows, but I think it's something like \emph{Programs\ra
Accessories\ra Command Prompt}. In the command prompt do the following:
\benum
	\item Change directory to where the \code{MassEmailer} jar program exists.
		I'll call it \code{massEmailer.jar}.
	\item Run the jar with java using its absolute path. It's probably something like:
		\code{"C:\textbackslash{}Program Files (x86)\textbackslash{}Java\textbackslash{}jre7\textbackslash{}bin\textbackslash{}java" -jar massiveEmailer.jar argument}
	\item \code{argument} is either 1 or 2, depending on which email type you want to send
\eenum

\ssn{Running using Linux}

Running on Linux is SOOOOO easy. You could open it like Windows: use your file
browser (dolphin in KDE and nautilus in GNOME). Right click the jar file, "Open
with", and then type \code{java -jar} in the text box. 

Another way is to run it in a terminal which I recommend because you'll be able
to view any errors that get created during the running of the program. Change
to the directory containing the jar file and run:
\bitem
	\citem{java -jar paymentValidator.jar}{}
	\citem{java -jar onsiteRegistration.jar}{}
	\citem{java -jar massEmailer.jar [argument]}{where \code{argument} is either 1 or 2}
\eitem

\ssn{Running using MacOS}

I have no idea. It's probably similar to Linux since they're both UNIX-like
operating systems. Google will be your friend.

\sn{Troubleshooting, Problems}

For the onsite worksheets for the early/late registration spreadsheets, copy
over the formsite header that is automatically generated for the formsite
worksheet.

Note that the \code{PaymentValidator} will throw a bunch of errors if you set
it to a worksheet that contains \tb{NO} registrants i.e. it is empty except for
the formsite question header.

\ssn{I double-click on the jar but nothing happens}

Make sure you hava Java installed. If that's not it, then it's probably
because you do not have all the configuration files that you need. If you run
the program in a terminal, it will produce output that will hopefully be able
to help you discover the problem and potentially which file you are missing.

The files you need are all found in the \code{files} folder of the source code.
I recommend distributing the \code{files} directory with the jar executable
files during the festival, but not with the source code. 

Also, make sure that the \code{files/config.xml} file exists. This should
\tb{never} be put in the repository but emailed/shared amongst the tango
members because is contains the google username and password of our account.

\sn{Hacking notes}

Note that according to Google, when you use the LineFeed to get information
from the spreadsheet, the first row will not be returned in that LineFeed.
This is because the first row is considered a header row and thus
contains no data. You can customize your query using a CellFeed query,
however, this takes more work on your part for processing the CellFeed.

\end{document}
